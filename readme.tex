\documentclass{article}
\usepackage[CJKnumber]{xeCJK}
\setCJKmainfont[BoldFont=SimHei]{SimSun}
\usepackage{url,hyperref}
\title{符合GB/T7714-2005标准的biblatex格式文件(v0.02b)}
\author{李志奇}
\begin{document}
\maketitle

\section{简介}
	这是我编写的基于biblatex的文献生成格式文件,目标是符合GB/T7714-2005标准。希望能对大家使用\LaTeX 有帮助。
	\subsection{文件说明}
	\begin{enumerate}
		\item \verb|./src/gbt7714_2005_n.bbx|和\verb|./src/gbt7714_2005_n.cbx|可以生成符合GB/T7714-2005中顺
			序索引规范的引用文献。
	
		\item \verb|./src/gbt7714_2005_ay.bbx|和\verb|./src/gbt7714_2005_ay.cbx|可以生成符合GB/T7714-2005中
			著者-出版年索引规范的引用文献。
	
		\item \verb|./src/gbt7714_2005.def|包含了两种引用方式都会用到的命令。
	
		\item \verb|./src/authoryear_test.tex|和\verb|./src/numeric_test.tex|分布为著者-出版年和顺序索引方式
			的测试文件,其中的注释包含了对于使用方法的说明。
	
		\item \verb|.src/ref.bib|中包含了测试中使用到的所有引用条目,绝大大多数来自GB/T7714-2005规范
	
		\item Makefile和run.bat为生成pdf文件的脚本文件
	\end{enumerate}
	
   \subsection{测试环境}
   \begin{enumerate}
		\item Ubuntu10.04LTS + xelatex(texlive2009) + xeCJK2.42 + biblatex1.2a +
			biber0.8.2,
		\item Windows7 + texlive2012(xelatex + xeCJK + biblatex + biber)。
	\end{enumerate}
	
\section{关于.bib文件}
	.bib文件中的域条目基本遵循biblatex中的定义,有3点需要说明
	\begin{enumerate}
		\item language域被用来区别中英文文献。如果是中文文献,language域一定要设置为cn或者zh。
			英文文献的language可以不设定。
		\item 对于连续出版物(periodical)的volume和number域进行了扩展,它们可以是形如``123-234''的区间值
		\item 由于GB/T7714-2005标准规定的文献类型标志在biblatex中没有对应项,所以biblatex的自定义域usera表示
	\end{enumerate}
\section{存在的问题}
	\begin{enumerate}
		\item 目前所有的标点都采用半角标点。这样是否正确还不清楚。是不是需要加入其它标点
			使用方案,例如英文文献使用半角标点,中文文献使用全角标点,又或者全使用全角比
			标点
		\item 目前文献项目的断行还需要进一步完善。
		\item 如果使用者在使用过程中发现任何问题或者有任何建议,请不吝赐教。我的邮箱地址是
		\url{lizhqi1983@163.com}
	\end{enumerate}
	
\section{ChangeLog}
\subsection{0.02b}
	\begin{enumerate}
	\item 添加了与Makefile对于的windows批处理文件run.bat
	\item 修正了readme.tex的一处疏漏
	\item 修正了\verb|authoryear_test.tex|中的一处错误
	\item 修正了在测试环境1下对英文名字处理的一处错误
	\end{enumerate}
\end{document}
