\documentclass{article}
\usepackage{fontspec}
\usepackage{xunicode}
\usepackage{xltxtra}
\usepackage[CJKnumber]{xeCJK}
%windows
\setCJKmainfont[BoldFont=SimHei]{SimSun}
%linux
%\setCJKmainfont[BoldFont=WenQuanYi Zen Hei]{AR PL UMing CN}
%支持的选项
%gbtverbose 在一些关键域空缺时(例如title等),则会在该域应该出现的位置打印警告信息(例如“标题空缺”)。默认为false
%lastnamefirst 英文名字姓在前名在后,例如author域设置为 Donald E. Knuth 那么输出中"Knuth Donald E.",默认为true
%gbtinits 名字的缩写采用GB/T7714中的规范,例如Donald E. Knuth的名字的缩写为 "D E Knuth" 而不是 "D. E. Knuth", 默认为true
%firstinits 名字的缩写采用加缩写符号的方式即Donald E. Knuth的名字的缩写为缩写为“D. E. Knuth",
%设定这一项为true的同时要把gbtinits设定为false

%设定gbinits=false,firstinits=true,lastnamefirst=false,可以输出符合一般英文名规
%范的输出,例如Donald E. Knuth 的输出为 D. E. Knuth
\usepackage[backend=biber,gbverbose=true,style=gb7714n,gbinits=true,
firstinits=true,lastnamefirst=true]{biblatex}
	
%重定义这个命令可以更改应用文献一节中采用的字体,该命令可以包含任何合法的Latex字体命令
\renewcommand{\bibfont}{\fontsize{8pt}{8pt}\selectfont}

%可以通过\defbibheading命令定义参考文献一节的章节标题的样式,
%在打印参考文献时使用heading选项选择想要的标题样式,例如\printbibliography[heading=nobibhead]
\newcommand*{\zhrefname}{参考文献}
\defbibheading{bibhead}[\zhrefname]{%
    \section*{#1}%
    \markboth{#1}{#1}}
	
\defbibheading{nobibhead}[]{}

%添加文献数据库
\addbibresource{./ref.bib}

%\usepackage{hyperref}
%debug
\tracingcommands=0
\tracingmacros=0
%\tracingoutput=1

\begin{document}
\section{专著(book)}\newrefsection%使用\newrefsession可以结束上一个引用章节,开始下一个,详细说明请参看biblatex文档
\nocite{book1}
\nocite{book2}
\nocite{book3}
\nocite{book4}
\nocite{book5}
\nocite{book6}
\nocite{book7}
\nocite{book8}
\nocite{book9}
\nocite{book10}
\nocite{book11}

%使用\cite命令引用文献,\cite命令的完全格式是\cite[prenote][postnote]{key}
%输出结果为“prenote[文献标号]postnote”。postenote可以用来指定GBT7714中要求的引用页码。
引用测试\cite[][200--300]{book1}

连续引用测试\cite{book1,book2,book3,book5}

连续引用测试2\cite{book1,book2,book3,book5,book6,book7, book11}

连续引用测试3\cite{book2,book3,book5,book6,book7, book11}
\printbibliography[heading=nobibhead]%使用\printbibliography命令打印参考文献,详细参数请参看biblatex文档

\section{专著中的析出文献(inbook)}\newrefsection
\nocite{inbook1}
\nocite{inbook2}
\nocite{inbook3}
\nocite{inbook4}
\nocite{inbook5}
\nocite{inbook6}
\printbibliography[heading=nobibhead]

\section{连续出版物(periodical)}\newrefsection
\nocite{periodical1}
\nocite{periodical2}
\nocite{periodical3}
\printbibliography[heading=nobibhead]

\section{连续出版物中的析出文献(article)}\newrefsection
\nocite{article1}
\nocite{article2}
\nocite{article3}
\nocite{article4}
\nocite{article5}
\nocite{article6}
\printbibliography[heading=nobibhead]

\section{专利文献(patent)}\newrefsection
\nocite{patent1}
\nocite{patent2}
\nocite{patent3}
\printbibliography[heading=nobibhead]

\section{电子文献(online)}\newrefsection
\nocite{online1}
\nocite{online2}
\nocite{online3}
\printbibliography[heading=nobibhead]

\section{论文集、会议录(proceedings)}\newrefsection
\nocite{aproceedings1}
\nocite{aproceedings2}
\nocite{aproceedings3}
\printbibliography[heading=nobibhead]

\section{会议论文(inproceedings)}\newrefsection
在标准中没有这个类型,但实际中很常见。
\nocite{inproceeding1}
\nocite{ay5}
\nocite{ay7}
\printbibliography[heading=nobibhead]

\section{科技报告(report)}\newrefsection
\nocite{areport1}
\nocite{areport2}
\printbibliography[heading=nobibhead]

\section{学位论文(thesis)}\newrefsection
\nocite{athesis1}
\nocite{athesis2}
\printbibliography[heading=nobibhead]


\section{其它测试}\newrefsection
\nocite{booknoauthor}
\nocite{booknolocation}
\nocite{booknopublisher}
\nocite{booknodate}
\nocite{booknopages}
\nocite{bookadd1}
\nocite{inproceeding1}
\nocite{articlemorenames}
\nocite{bookmoretranslators}
\printbibliography[heading=bibhead]
\end{document}
