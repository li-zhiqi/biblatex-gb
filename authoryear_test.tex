\documentclass{article}
\usepackage{fontspec}
\usepackage{xunicode}
\usepackage{xltxtra}
\usepackage[CJKnumber]{xeCJK}
%windows
\setCJKmainfont[BoldFont=SimHei]{SimSun}
%linux
%\setCJKmainfont[BoldFont=WenQuanYi Zen Hei]{AR PL UMing CN}
%gb7714ay为作者出版年应用格式

%支持的选项
%gbtverbose 在一些关键域空缺时(例如title等),则会在该域应该出现的位置打印警告信息(例如“标题空缺”)。默认为false
%lastnamefirst 英文名字姓在前名在后,例如author域设置为 Donald E. Knuth 那么输出中"Knuth Donald E.",默认为true
%gbtinits 名字的缩写采用GB/T7714中的规范,例如Donald E. Knuth的名字的缩写为 "D E Knuth" 而不是 "D. E. Knuth", 默认为true
%firstinits 名字的缩写采用加缩写符号的方式即Donald E. Knuth的名字的缩写为缩写为“D. E. Knuth",
%设定这一项为true的同时要把gbtinits设定为false

%设定gbtinits=false,firstinits=true,lastnamefirst=false,可以输出符合一般英文名规
%范的输出,例如Donald E. Knuth 的输出为 D. E. Knuth

\usepackage[backend=biber,gbverbose=true,sorting=nty, style=gb7714ay]{biblatex}

%重定义这个命令可以更改应用文献一节中采用的字体,该命令可以包含任何合法的Latex字体命令
\renewcommand{\bibfont}{\fontsize{8pt}{8pt}\selectfont}

%可以通过\defbibheading命令定义参考文献一节的章节标题的样式,
%在打印参考文献是使用heading选项选择想要的标题样式,例如\printbibliography[heading=nobibhead]
\newcommand*{\zhrefname}{参考文献}
\defbibheading{bibhead}[\zhrefname]{%
    \section*{\fontsize{8pt}{8pt}\selectfont#1}%
    \markboth{#1}{#1}}
	
\defbibheading{nobibhead}[]{}

\addbibresource{./ref.bib}

%debug
\tracingcommands=1
\tracingmacros=1
\tracingoutput=1
\begin{document}
\section{引用单篇文献}\newrefsection
使用\verb|\cite|生成引用,\verb|\cite*|生成不包含作者名的引用,示例如下

The notion of an invisible college has been explored in the sciences\cite{ay1}. Its absence among historians is notes by
Stieg\cite*{ay2}\dots.
\printbibliography[heading=bibhead]
\section{引用同一著者同年出版的多篇文献}\newrefsection%使用\newrefsession可以结束上一个引用章节,开始下一个,详细说明请参看biblatex文档
\nocite{ay3}
\nocite{ay4}
\printbibliography[heading=bibhead]%使用\printbibliography命令打印参考文献,详细参数请参看biblatex文档
\section{多次引用同一著者的同一文献}\newrefsection
使用\verb|\cite[][pages]{}|生成包含引用页数的引用,示例如下:

\nocite{ay5}\nocite{ay6}\nocite{ay7}\nocite{ay8}
主编靠编辑思想指挥全局已是编辑界的共识\cite{ay7},然而对于编辑思想至今没有一个明确的界定,故不妨提出一个架构\dots\dots 参与
讨论。由于“思想”的内涵是“客观存在反映在人的意识中经过思维活动而产生的结果”\cite[][1194]{ay8},所以“编辑思想”的内涵就是编辑实践
反映在编辑工作的意识中“经过思维活动而产生的结果”。\dots\dots《中国青年》杂志创办人追求的高格调——理性的成熟与热点的凝聚\cite{ay6},
表明其读者群的文化品位的高层次\dots\dots“方针”指“引导事业前进的方向和目标”\cite[][354]{ay8}。\dots\dots 对编辑方针,1981年中国科协
副主席裴丽生曾经有过科学的论断——“自然科学学术期刊必须坚持以马列主义、毛泽东思想为指导,贯彻为国民经济发展服务,理论和实践相结合,普及
与提高相结合,‘百花齐放,百家争鸣’的方针”\cite{ay5}它完整地回答了为谁服务,怎样服务,如何服务的更好的问题。
\printbibliography[heading=bibhead]
\section{中英文文献排序测试}\newrefsection
\nocite{ay1}\nocite{ay2}\nocite{ay5}\nocite{ay6}\nocite{ay3}\nocite{ay4}\nocite{ay7}\nocite{ay8}
\printbibliography[heading=bibhead]
\end{document}

